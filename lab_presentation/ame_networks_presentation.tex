\documentclass[newPxFont,numfooter,sectionpages]{beamer}
\usepackage[utf8]{inputenc}
\usetheme{sthlm}
\usepackage[round]{natbib}
\bibliographystyle{plainnat}
\usepackage{amsmath}
\usepackage{amssymb}
\usepackage{amsthm}
\usepackage{dsfont}
\usepackage{hyperref}
\usepackage{multirow}
\usepackage{color}
\usepackage{tikz}
\usepackage{graphicx}
\usepackage{setspace}
\usepackage{bigints}
\usepackage{algorithm2e}
\usepackage{float}
\usepackage{lscape}
\usepackage{rotating}
\usepackage{longtable}
\usepackage[normalem]{ulem}
\hypersetup{
    colorlinks=true,
    linkcolor=black,
    filecolor=magenta,
    urlcolor=blue,
    citecolor=black,
}

\newcommand{\indep}{\mathrel{\text{\scalebox{1.07}{$\perp\mkern-10mu\perp$}}}}
\DeclareMathOperator*{\argmin}{arg\,min}
\DeclareMathOperator*{\argmax}{arg\,max}
\newcommand{\E}{\mathbb{E}}
\newcommand{\V}{\mathbb{V}}
\newcommand{\I}{\mathbb{I}}
\newcommand{\com}[1]{&&\mbox{(#1)}}
\newcommand{\bt}{\mathbf{t}}
\newcommand{\bT}{\mathbf{T}}
\newcommand{\Gni}{G_{\mathcal{N}_i^\bt}}
\newcommand{\Gnj}{G_{\mathcal{N}_j^\bt}}
\newcommand{\btheta}{\boldsymbol{\phi}}

\title{Almost-Matching-Exactly for Treatment Effect Estimation under Network Interference}
\subtitle{}
\author{Usaid Awan, Marco Morucci, Vittorio Orlandi}

%\institute{Duke University}
\date{}
\begin{document}
\setbeamercolor{background canvas}{bg=dukeBlue}
\setbeamercolor{title}{fg=white} \setbeamercolor{subtitle}{fg=white}
\setbeamercolor{institute}{fg=white}
\setbeamercolor{author}{fg=white}
\setbeamercolor{normal text}{fg=white}
\maketitle

\setbeamercolor{background canvas}{bg=white}
\setbeamercolor{frametitle}{bg=dukeBlue}
\setbeamercolor{normal text}{fg=sthlmDarkGrey, bg=sthlmGrey}

\begin{frame}{Setting - Causal Inference}
\begin{itemize}
  \item We have $i=1, \dots, n$ experimental units
  \item Treatment $t_i \in \{0, 1\}$ with $\bt \in \{0, 1\}^n$ is a binary vector with the treatment level of every unit.
  \item Potential outcomes $Y_i(t_i, \bt)$ are random variables and depend on both treatment of unit $i$ (1st argument), and treatment of \textbf{all other units} (2nd argument).
  \item Observed treatment $\bT \in \{0, 1\}^n$ is assigned \textbf{uniformly at random}.
  \item Observed outcome: $Y_i = T_iY(1, \bT) + (1-T_i)Y_i(0, \bT)$
  \begin{itemize}
    \item Since treatment is randomized: $\E[Y_i|\bT = \bt, T_i = t] = \E[Y_i(t, \bT)]$ (Ignorability).
  \end{itemize}
  \item Units are connected in a network, $G$, in which unit $i$'s \textbf{treated neighborhood subgraph} is $\Gni$.
\end{itemize}
\end{frame}

\begin{frame}{The Problem: No SUTVA!}
Usually we assume SUTVA: that \textit{units' treatments don't influence other units' outcomes}, but we can't do that here because our units are connected in a network:
\begin{center}
\includegraphics[height=1.8in]{graph1.png}
\end{center}
It could be that the treatment assigned to $j$ influences the outcome of $i$ through their connection in the network.
\end{frame}

\begin{frame}{Similar Graphs Carry Similar Interference}
\begin{alertblock}{Idea}
What if the amount of interference experienced by a unit depended on the \textit{shape} of its treated neighborhood subgraph?
\end{alertblock}
\begin{center}
\includegraphics[height=1.5in]{graph2.png}
\end{center}
Then, in expectation, two units with the same treated neighborhood graph will respond similarly to the treatment. \\
\textbf{We can use this idea to do matching to reduce interference.}
\end{frame}

\begin{frame}{Assumptions}
\begin{enumerate}
  \item Outcome model: $Y_i = \alpha + t_i\beta_i + f(\Gni) + \epsilon_i$
  \begin{itemize}
    \item $f$ is some interference function dependent on $\Gni$, the \textbf{treated neighborhood subgraph} of unit $i$.
  \end{itemize}
  \item $\E[\epsilon_i] = 0$
    \begin{itemize}
    \item Ignorability
  \end{itemize}
  \item $|\E[f(g)] - \E[f(h)]| \leq K_1||g - h||$
  \begin{itemize}
    \item The more similar the neighborhood graphs of $i$ and $j$, the more similar the amount of interference they receive.
    \item Together with (1), this assumption encodes a version of SANASIA (Airoldi and Sussman, 2018) conditional on unit's neighborhood subgraphs.
  \end{itemize}
\end{enumerate}
\begin{alertblock}{Problem}
How do we represent similarity between neighborhood graphs?
\end{alertblock}
\end{frame}
\begin{frame}{Subgraph Counts}
	\begin{itemize}
		\item We'll say that neighborhood graphs are similar if they contain similar counts of subgraphs 
		\item What subgraphs? How similar must the counts be? How similar does this make the graphs?
		\item Use FLAME to decide
	\end{itemize}
\end{frame}
\begin{frame}{FLAME: An Overview}
	\begin{itemize}
		\item FLAME (Fast Large-Scale Almost Matching Exactly) is a method for creating interpretable matches between units with discrete covariates
		\begin{enumerate}
			\item Match units exactly on as many covariates as possible
			\item Drop a covariate
			\item Repeat
		\end{enumerate}
		\item At each step, drop the covariate maximizing match quality:
		\[\mathtt{MQ} = C \cdot \mathtt{BF} - \mathtt{PE}\]
		\item $\mathtt{BF} = $ prop. controls matched + prop. treated matched
		\item $\mathtt{PE} = $ prediction error achieved by remaining covariates
		\item Tradeoff between making matches and accurate prediction
	\end{itemize}
\end{frame}
\begin{frame}{$\cdot$AME-Networks}
\begin{alertblock}{Problem \#2}
How do we choose which subgraphs we should use to represent the treated neighborhood graphs of our units?
\end{alertblock}
\textbf{Use FLAME to choose which subgraphs to count!}
\begin{enumerate}
	\item Enumerate (up to isomorphism) all $p$ subgraphs $S_1, \dots, S_p$ seen across all the $\mathcal{N}_i$, i = 1, \dots, n
  \item For each unit $i$, count how many of the $S_j$ are in $\mathcal{N}_i$
  \item These are likely a lot (maximum on the order of $|\mathcal{N}_i|^2$) and it's unlikely that many units will have identical counts
  \item Use FLAME to make almost-exact matches on subgraph counts 
\end{enumerate}
\end{frame}
\begin{frame}{A Small Change to FLAME}
We know from our theoretical setup that network statistics should do two things well:
\begin{enumerate}
  \item Predict the \textbf{outcomes}
  \item Predict the \textbf{network}
\end{enumerate}
Therefore, it makes sense that the FLAME objective should trade-off between these things:
\begin{align*}
{\tt PE} &= \sum_{t = 0}^1\argmin_{f \in \mathcal{F}}\frac{1}{n}\sum_{i = 1}^n(Y_i - f(S(\Gni)))^2\\
         &- D\argmax_{\btheta \in \mathbb{R}^{|S|_0}}\underbrace{\frac{1}{n}\sum_{i=1}^n\btheta^TS(\Gni) - \log\left(\sum_{g \in \mathcal{G}} \btheta^TS(g)\right)}_{\text{ERGM log-likelihood}}
\end{align*}
\end{frame}
\begin{frame}{(OLD) Simulation Results}
\centering
\includegraphics[height=3in]{simple.png}
\end{frame}
\begin{frame}{(OLD) Simulation Results}
\centering
\includegraphics[height=3in]{even_more_complex.png}
\end{frame}
\begin{frame}{(OLD) Simulation Results}
\centering
\includegraphics[height=3in]{unrelated_dense.png}
\end{frame}
\begin{frame}{Bias Bound for oracle AME}
As a preliminary result we can say that, under all the assumptions made before, with the true value of $\btheta$ and $S$ known, and if we choose a match for unit $i$ with treated neighborhood graph $g$ such that:
\begin{align*}
j \in {\tt MG(g)} \text{ if } j \in \argmin_{j = 1, \dots, n, T_j = 0} |\btheta^TS(g) - \btheta^TS(\Gnj)|,
\end{align*}
then the bias for the CATT of $i$ can be upper bounded by:
\begin{align*}
|\E[Y_i - Y_j] - \tau_i| &\leq K_1\sum_{h \in \mathcal{G}}|\btheta^TS(g) - \btheta^TS(h)|\frac{exp(\btheta^TS(h))}{\sum_{\ell \in \mathcal{G}}exp(\btheta^TS(\ell))}\\
&\times\left[\sum_{d = S(g) - |S(g) - S(h)|}^{S(g) + |S(g) - S(h)|}\frac{D_{\btheta, S}(d)exp(d)}{\sum_{\ell \in \mathcal{G}}exp(\btheta^TS(\ell))}\right]^{n - 1}
\end{align*}
\end{frame}
\begin{frame}{Plan: We Want to Hit the October 8 AISTATS Deadline}
\begin{enumerate}
  \item We need to implement the revised $\tt PE$ function
  \item We need to redo the simulations
  \item More theory? Statements on how well the subgraph count can ``encode'' a given graph would be nice
  \item Find a good application
  \item Write!
\end{enumerate}
\end{frame}
\end{document}




